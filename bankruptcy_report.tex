\documentclass{article}

\usepackage{fancyhdr}
\usepackage{extramarks}
\usepackage{amsmath}
\usepackage{amsthm}
\usepackage{amsfonts}
\usepackage{tikz}
\usepackage[plain]{algorithm}
\usepackage{algpseudocode}
\usepackage{bbm}
\usepackage{hyperref}

%
% For code output
% From: https://www.overleaf.com/learn/latex/Code_listing
%
\usepackage{listings}
\usepackage{xcolor}

\definecolor{codegreen}{rgb}{0,0.6,0}
\definecolor{codegray}{rgb}{0.5,0.5,0.5}
\definecolor{codepurple}{rgb}{0.58,0,0.82}
\definecolor{backcolour}{rgb}{0.95,0.95,0.92}

\lstdefinestyle{mystyle}{
    backgroundcolor=\color{backcolour},   
    commentstyle=\color{codegreen},
    keywordstyle=\color{magenta},
    numberstyle=\tiny\color{codegray},
    stringstyle=\color{codepurple},
    basicstyle=\ttfamily\footnotesize,
    breakatwhitespace=false,         
    breaklines=true,                 
    captionpos=b,                    
    keepspaces=true,                 
    numbers=left,                    
    numbersep=5pt,                  
    showspaces=false,                
    showstringspaces=false,
    showtabs=false,                  
    tabsize=2
}

\lstset{style=mystyle}

\usetikzlibrary{automata,positioning}
%
% Basic Document Settings
%

\topmargin=-0.45in
\evensidemargin=0in
\oddsidemargin=0in
\textwidth=6.5in
\textheight=9.0in
\headsep=0.25in

\linespread{1.1}

\pagestyle{fancy}
\lhead{\hmwkAuthorName}
\chead{\hmwkClass\ : \hmwkTitle}
\rhead{\firstxmark}
\lfoot{\lastxmark}
\cfoot{\thepage}

\renewcommand\headrulewidth{0.4pt}
\renewcommand\footrulewidth{0.4pt}

\setlength\parindent{0pt}

%
% Create Problem Sections
%

\newcommand{\enterProblemHeader}[1]{
    \nobreak\extramarks{}{Problem \arabic{#1} continued on next page\ldots}\nobreak{}
    \nobreak\extramarks{Problem \arabic{#1} (continued)}{Problem \arabic{#1} continued on next page\ldots}\nobreak{}
}

\newcommand{\exitProblemHeader}[1]{
    \nobreak\extramarks{Problem \arabic{#1} (continued)}{Problem \arabic{#1} continued on next page\ldots}\nobreak{}
    \stepcounter{#1}
    \nobreak\extramarks{Problem \arabic{#1}}{}\nobreak{}
}

\setcounter{secnumdepth}{0}
\newcounter{partCounter}
\newcounter{homeworkProblemCounter}
\setcounter{homeworkProblemCounter}{1}
\nobreak\extramarks{Problem \arabic{homeworkProblemCounter}}{}\nobreak{}

%
% Homework Problem Environment
%
% This environment takes an optional argument. When given, it will adjust the
% problem counter. This is useful for when the problems given for your
% assignment aren't sequential. See the last 3 problems of this template for an
% example.
%
\newenvironment{homeworkProblem}[1][-1]{
    \ifnum#1>0
        \setcounter{homeworkProblemCounter}{#1}
    \fi
    \section{Problem \arabic{homeworkProblemCounter}}
    \setcounter{partCounter}{1}
    \enterProblemHeader{homeworkProblemCounter}
}{
    \exitProblemHeader{homeworkProblemCounter}
}

%
% Homework Details

%   - Title
%   - Submission date
%   - Course code
%   - Author
%

\newcommand{\hmwkTitle}{Project 2 - Bankruptcy Report}
\newcommand{\hmwkDueDate}{March 23, 2023, University of Toronto}
\newcommand{\hmwkClass}{STA 2453}
\newcommand{\hmwkAuthorName}{\textbf{Vignesh Edithal}}

%
% Title Page
%

\title{
    \vspace{2in}
    \textmd{\textbf{\hmwkClass:\ \hmwkTitle}}\\
    \normalsize\vspace{0.1in}\small{\hmwkDueDate}\\
    \vspace{3in}
}

\author{\hmwkAuthorName}
\date{}

\renewcommand{\part}[1]{\textbf{\large Part \Alph{partCounter}}\stepcounter{partCounter}\\}


\begin{document}

\maketitle

\pagebreak

\begin{homeworkProblem}

\textbf{Model development scheme and training}

\vspace{10px}

Data pre-processing

\begin{itemize}
    \item City and Province are concatenated to get Address column which is then frequency encoded such that top 9 frequent addresses are mapped from 1 to 9 and others 10.
    \item AmountOfInsurance and Earnings are transformed by first dividing by 1000 then taking log base 10.
    \item BusinessType is encoded based on categorical codes.
    \item The 250k rows of policies data with the above transformation are clustered into 12 categories (corresponding to best BIC score) using GaussianMixture API from sklearn.
    \item Policies data and Claims data are merged using outer join on BusinessID.
    \item Rows are de-duplicated and NumClaims and ExpectedSeverity columns are added by aggregating over duplicate BusinessID rows.
\end{itemize}

\vspace{10px}

NumClaims Model

\begin{itemize}
    \item All 250k rows are considered for training data, NumClaims for businesses which did not claim is set to 0.
    \item XGBRegressor API from sklearn was fit on training data with 5 fold cross validation and grid search for hyperparameter tuning.
    \item Best hyperparameters found, n\_estimators: 200, learning\_rate: 0.1, max\_depth: 20, subsample: 0.6, colsample\_bytree: 0.95
\end{itemize}

\vspace{10px}

ExpectedSeverity Model

\begin{itemize}
    \item Only 8472 rows are considered for training data, these are the business which have claimed at least once.
    \item XGBRegressor API from sklearn was fit on training data with 5 fold cross validation and grid search for hyperparameter tuning.
    \item Best hyperparameters found, n\_estimators: 200, learning\_rate: 0.1, max\_depth: 20, subsample: 0.6, colsample\_bytree: 0.95
\end{itemize}

Inference

\begin{itemize}
    \item ExpectedSeverity and NumClaims are predicted and plugged into the Loss Cost formula with given constants
    \item ProfitLoading which can range from 0 to 0.6 was set to a value of 0.3
\end{itemize}

\end{homeworkProblem}

\pagebreak

\begin{homeworkProblem}

\textbf{Model performance statistics}

\vspace{10px}

\begin{itemize}
    \item Out of 2721 customers, I was awarded 577 (around 25\%) customers. This is a lot of customers given there were 18 competitors in the market
    \item The highest loss I have endured is \$559k
    \item I have endured 67 losses and the rest customers resulted profit. However, the total loss is \$726k
\end{itemize}

\end{homeworkProblem}

\pagebreak

\begin{homeworkProblem}

\textbf{Future Direction and Business Strategies}

\vspace{10px}

Business Strategies

\begin{itemize}
    \item It is a good idea to overcharge to minimize risk.
    \item The goal of insurance premium modelling is to minimize risk as much as possible.
    \item One way of minimizing risk is to manually increase the premium of risky customers (customers with expected severity over a certain threshold)
\end{itemize}

\vspace{10px}

Future directions to prevent bankruptcy

\begin{itemize}
    \item I increased my profit loading from 0.3 to 0.5 and generated new premiums.
    \item The premiums csv was downloaded and my premiums were updated to new values.
    \item dirchlet\_market.py script was run with new market values values to identify which customers got awarded to me.
    \item Upon analysis of profit it was found that I made a profit of \$1,003,871
\end{itemize}

\end{homeworkProblem}

\end{document}
